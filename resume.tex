\documentclass{article}

\usepackage{fullpage}
\usepackage{amsmath}
\usepackage{amssymb}
\usepackage[T1]{fontenc}
\usepackage{fancyhdr}
\usepackage{lastpage}
\usepackage{graphicx}
\usepackage{fontawesome}
\usepackage{setspace}
\usepackage[usenames,dvipsnames]{xcolor}
\usepackage[colorlinks=true, urlcolor=ColorTwo]{hyperref}
\usepackage{datetime}
\usepackage{lipsum}

\newcommand\blfootnote[1]{%
  \begingroup
  \renewcommand\thefootnote{}\footnote{#1}%
  \addtocounter{footnote}{-1}%
  \endgroup
}

% DEFINITIONS FOR RESUME

\textheight=10in
\pagestyle{fancy}
\raggedright
\fancyhf{}
\renewcommand{\headrulewidth}{0pt}

\setlength{\hoffset}{-2pt}
\setlength{\footskip}{20pt}

\def\bull{\vrule height 0.8ex width .7ex depth -.1ex }

\newcommand{\contact}[3]{
\vspace*{5pt}
\begin{center}
{\LARGE \scshape {#1}}\\
\vspace{3pt}
#2 
\vspace{2pt}
#3
\end{center}
\vspace*{-8pt}
}

\newcommand{\header}[1]{{
\hspace*{-15pt}\vspace*{6pt} \textsc{#1}} \vspace*{-6pt} 
\lineunder
}

\newcommand{\lineunder}{
\vspace*{-8pt} \\ \hspace*{-18pt} 
\hrulefill \\
}

\newcommand{\content}{
\vspace*{2pt}%
}

\newcommand{\college}[5]{\vspace*{2pt}% 
\textbf{#1} \labelitemi #2 \labelitemi #3 \hfill #4 \\ #5 
\vspace*{5pt}
}

\newcommand{\school}[4]{
\textbf{#1} \labelitemi #2 \hfill #3 \\ #4 \vspace*{5pt}
}

\newcommand{\employer}[4]{{
\vspace*{2pt}%
\textbf{#1} #2 \hfill #3\\ #4 \vspace*{2pt}}
}

\newcommand{\project}[3]{{
\vspace*{2pt}% 
\textbf{#1} \hfill #2\\ #3 \vspace*{2pt}}
}

\renewcommand{\labelitemi}{
$\vcenter{\hbox{\tiny$\bullet$}}$\hspace*{3pt}
}

\renewcommand{\labelitemii}{
$\vcenter{\hbox{\tiny$\bullet$}}$\hspace*{-3pt}
}

%
% Color theme
%
\definecolor{ColorOne}{RGB}{0,110,140} 	% Blue
\definecolor{ColorTwo}{RGB}{120,0,120} 	% Mauve
%
% Format hyperrefs
%
\newcommand{\myhref}[2]{%
\href{#1}{\textcolor{ColorTwo}{#2}}
}

\newenvironment{bullet-list-major}{
\begin{list}{\labelitemii}{\setlength\leftmargin{3pt} 
\topsep 0pt \itemsep -2pt}}{\vspace*{4pt}\end{list}
}

\newenvironment{bullet-list-minor}{
\begin{list}{\labelitemii}{\setlength\leftmargin{15pt} 
\topsep 0pt \itemsep -2pt}}{\vspace*{4pt}\end{list}
}

\cfoot{
Last updated: July 7, 2018
}

\rfoot{
Page $\thepage\hspace*{3pt}\vert\hspace*{3pt}\pageref{LastPage}$
}

% END RESUME DEFINITIONS

\begin{document}

\small
\smallskip
\vspace*{-44pt}

\contact{Lei Wang}
{
\textcolor{ColorTwo}{\faGithub} 
\myhref{https://github.com/LeiWang1999}{LeiWang1999} 
\labelitemi 
\textcolor{ColorTwo}{\faEnvelopeO} 
\myhref{mailto:leiwang1999@outlook.com}{leiwang1999@outlook.com}
\labelitemi
\textcolor{ColorTwo}{\faChain} 
\myhref{https://leiblog.wang}{leiblog.wang}
}

\vspace{15pt}
\header{Education}
    \school{University of Chinese Academy of Science}{Beijing, China}{August 2021 -- Present}
    {\textit{Masters \labelitemi Computer Science}}

    \school{Nanjing Tech University}{Nanjing, China}{August 2017 -- June 2021}
    {\textit{Bachelor \labelitemi Electronic Engineering} \labelitemi \textcolor{ColorOne}{\textbf{Overall GPA: 3.95/4.0}}}

\vspace*{4pt}%
\header{Work Experience}
    \employer{Microsoft Research Asia}{-- System Research Intern}{April 2022 -- Present}{Beijing, China}
	\begin{bullet-list-minor}
	\item Advised by \myhref{https://xysmlx.github.io/}{Dr. Lingxiao Ma} and \myhref{https://www.microsoft.com/en-us/research/people/jxue/}{Dr. Jilong Xue}
	\item Maintaining Microsoft DNN compiler \myhref{https://github.com/microsoft/nnfusion}{NNFusion}.
	\item Research focus: Sparse Tensor Compilation, Auto Tensorize, LLM inference foundation.   
    \end{bullet-list-minor}

    \employer{Netease}{-- NPU Development Intern}{Sep. 2021 -- Oct. 2021}{Beijing, China}
	\begin{bullet-list-minor}
	\item NVDLA FPGA deployment and Software stack remapping.
    \end{bullet-list-minor}

\vspace*{4pt}%
\header{Projects}
    \project{Ladder \textcolor{ColorOne}
{waiting for publication}}{2023}{}
    \begin{bullet-list-minor}
	\item Tensor Code Generation for Accelerator. obtaining comparable performance with cuBLAS/Cutlass and outperforms tensorrt of tensor core program. Supporting diverse tensor format remapping.
    \end{bullet-list-minor}
    \project{AutoGPTQ.tvm     \textcolor{ColorTwo}{\faGithub} 
\myhref{https://github.com/LeiWang1999/AutoGPTQ.tvm}{code}}{2023}{}
    \begin{bullet-list-minor}
	\item tvm inference kernel for GPTQ.
    \end{bullet-list-minor}
    \project{Full Stack FPGA Implementation of NVDLA}{2021}{
    \textcolor{ColorTwo}{\faGithub} 
\myhref{https://github.com/LeiWang1999/ZYNQ-NVDLA}{Code Archive}
    \labelitemi
    \textcolor{ColorTwo}{\faBook} 
\myhref{https://zhuanlan.zhihu.com/p/378202360}{post:DLA Deploy}
    \labelitemi
        \textcolor{ColorTwo}{\faBook} 
    \myhref{https://zhuanlan.zhihu.com/p/401943271}{post:Compiler Design}
}
	\begin{bullet-list-minor}
	\item Full-stack FPGA implementation of NVDLA. To enhance the utility of this accelerator, we designed a new compiler and runtime to allow networks auto fallback between CPU and hardware
                accelerators.
    \end{bullet-list-minor}
    \project{FPGA Accelerator for Beam Forming}{2020}{
    \textcolor{ColorTwo}{\faVideoCamera} 
    \myhref{https://github.com/LeiWang1999}{Demo}
}
	\begin{bullet-list-minor}
	\item FPGA acceleration to enhance sounds from specific points with tetragonal microphone array.
    \end{bullet-list-minor}
    \project{FPGA Accelerator for Digital Recognition}{2020}{
    \textcolor{ColorTwo}{\faVideoCamera} 
    \myhref{https://github.com/LeiWang1999}{Demo}
}
	\begin{bullet-list-minor}
	\item This project aims to provide accelerated digital analysis with lenet5.
    \end{bullet-list-minor}
    \project{Opensource Contributions     \textcolor{ColorTwo}{\faGithub} 
    \myhref{https://github.com/LeiWang1999}{LeiWang1999}}{-}{
}
	\begin{bullet-list-minor}
	\item gptq-integration to mlc-llm, matrix core support for tvm, general n:m training for apex, etc.
    \end{bullet-list-minor}
 
\vspace*{4pt}%
\header{Publications}
    \begin{bullet-list-major}
    \item Lin Bin*; Zheng Ningxin*; \textbf{Wang Lei}*; Cao Shijie; Ma Lingxiao; Zhang Quanlu; Zhu Yi; Cao Ting; Xue Jilong; Yang Yuqing; et al. \textbf{Efficient GPU Kernels for N: M-Sparse Weights in Deep Learning}. \textit{Proceedings of Machine Learning and Systems}, Vol. 5, 2023. (* represents co-first author) \hspace{2pt} \textcolor{ColorTwo}{\faLink}  \myhref{https://proceedings.mlsys.org/paper_files/paper/2023/file/4552cedd396a308320209f75f56a5ad5-Paper-mlsys2023.pdf}{Read Paper}
    \vspace{2pt}
    \item Sun Xiaotian; Wang Xinyu; Li Wanqian; \textbf{Wang Lei}; Han Yinhe; Chen Xiaoming. \textbf{PIMCOMP: A Universal Compilation Framework for Crossbar-based PIM DNN Accelerators}. \textit{60th. Design Automation Conference}, 2023.    \textcolor{ColorTwo}{\faLink}  \myhref{https://arxiv.org/pdf/2307.01475.pdf}{Read Paper}
    \vspace{2pt}
    \item \textbf{Lei Wang}; Lingxiao Ma; Shijie Cao; Ningxin Zheng; Quanlu Zhang. \textbf{Ladder: Efficient Tensor Compilation on Customized Data Format}. \textit{17th USENIX Symposium on Operating Systems Design and Implementation (Poster)}, 2023.

    \end{bullet-list-major}

\vspace*{4pt}%
\header{Awards}
\begin{bullet-list-major}
    \item \textbf{2018 Chinese National Scholarship (Top 0.3\%)}
    \vspace{2pt}
    \item 2021 Excellent New Student Award of Chinese Academy of Science
    \vspace{2pt}
    \item \textbf{Njtech \textbf{Person of Year 2020}}
    \vspace{2pt}
    \item \textbf{First Price of 2019 NUEDC (Top 0.5\%)}
    \vspace{2pt}
    \item Third Price of Integrated Circuit Innovation Competition \href{http://leiblog.wang/FPGA车牌识别/}{(FPGA hardware Accelerator for digital recognition)}
    \vspace{2pt}
    \item Third prize of National FPGA Competition \href{http://leiblog.wang/ZYNQ声源定位波束形成/}{(FPGA based FOSDA Alogrithom Implementation)}
\end{bullet-list-major}

\blfootnote{
\textcolor{black}{Last updated: \monthname,~\the\year.} \vspace{1mm}
}

\end{document}
